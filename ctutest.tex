% arara: pdflatex: { synctex: yes }
% arara: makeindex: { style: ctuthesis }
% arara: bibtex

% The class takes all the key=value arguments that \ctusetup does,
% and a couple more: draft and oneside
\documentclass[twoside]{ctuthesis}

\DeclareUnicodeCharacter{2212}{-}
\usepackage[ruled,vlined]{algorithm2e}
\usepackage{float}

%\documentclass{article}
\renewenvironment{description}[1][0pt]
  {\list{}{\labelwidth=0pt \leftmargin=#1
   \let\makelabel\descriptionlabel}}
  {\endlist}

\parindent=0pt

\ctusetup{
% 	preprint = \ctuverlog,
	mainlanguage = english,
	titlelanguage = english,
% 	mainlanguage = czech,
% 	otherlanguages = {czech,english},
	title-czech = {MDP algoritmy v POMDPs.jl},
	title-english = {MDP algorithms in POMDPs.jl},
	subtitle-czech = {Zejména Finite-Horizon MDP},
	subtitle-english = {Finite-Horizon MDPs in particular},
	doctype = B,
	faculty = F3,
	department-czech = {Katedra Počítačů},
	department-english = {Department of Computer Science},
	author = {Tomáš Omasta},
	supervisor = {Ing. Jan Mrkos},
	supervisor-address = {E-325, \\ Charles Square 13, \\ Prague 2},
% 	supervisor-specialist = {John Doe},
	fieldofstudy-english = {Open Informatics},
	subfieldofstudy-english = {Artificial Intelligence and Computer Science},
	fieldofstudy-czech = {Otevřená Informatika},
	subfieldofstudy-czech = {Základy umělé inteligence a počítačových věd},
	keywords-czech = {Finite-Horizon, MDP, Julia},
	keywords-english = {Finite-Horizon, MDP, Julia},
	day = 10,
	month = 1,
	year = 2021,
	specification-file = {ctutest-zadani.pdf},
% 	front-specification = true,
	front-list-of-figures = false,
	front-list-of-tables = false,
%	monochrome = true,
%	layout-short = true,
}

\ctuprocess

\addto\ctucaptionsczech{%
	\def\supervisorname{Vedoucí}%
	\def\subfieldofstudyname{Studijní program}%
}

\ctutemplateset{maketitle twocolumn default}{
	\begin{twocolumnfrontmatterpage}
		\ctutemplate{twocolumn.thanks}
		\ctutemplate{twocolumn.declaration}
		\ctutemplate{twocolumn.abstract.in.titlelanguage}
		\ctutemplate{twocolumn.abstract.in.secondlanguage}
% 		\ctutemplate{twocolumn.tableofcontents}
% 		\ctutemplate{twocolumn.listoffigures}
	\end{twocolumnfrontmatterpage}
}

% Theorem declarations, this is the reasonable default, anybody can do what they wish.
% If you prefer theorems in italics rather than slanted, use \theoremstyle{plainit}
\theoremstyle{plain}
\newtheorem{theorem}{Theorem}[chapter]
\newtheorem{corollary}[theorem]{Corollary}
\newtheorem{lemma}[theorem]{Lemma}
\newtheorem{proposition}[theorem]{Proposition}

\theoremstyle{definition}
\newtheorem{definition}[theorem]{Definition}
\newtheorem{example}[theorem]{Example}
\newtheorem{conjecture}[theorem]{Conjecture}

\theoremstyle{note}
\newtheorem*{remark*}{Remark}
\newtheorem{remark}[theorem]{Remark}

\setlength{\parskip}{5ex plus 0.2ex minus 0.2ex}

% Abstract in Czech
\begin{abstract-czech}
Do povědomí vědecké komunity se začíná dostávat nový programovací jazyk, Julia. Díky svému prudkému nárustu uživatelské báze a svému prostředí postavenému na open-source knihovnách, se dá očekávat, že se stane jedním z nejpopulárnějších jazyků. V této práci představíme Julii, MDP, teorii s nimi spjatou spolu s možnými přístupy k jejich řešení. Postupně ukážeme jejich výhody i nevýhody a nakonec se dostaneme k Finite-Horizon MDP, jako jedné z nejpokročilejších metod. Navrhneme design Finite-Horizon MDP rozhraní na základě koncepce z JuliaPOMDP/FiniteHorizonPOMDPs.jl \cite{FHPOMDP}, implementujeme jej a porovnáme s value iteration řešením DiscreteValueIteration.jl \cite{DVI} implementovaným v knihovně JuliaPOMDP's \cite{JuliaPOMDP}.
\end{abstract-czech}

% Abstract in English
\begin{abstract-english}
In the last couple of years, a new programming language, Julia, has emerged into the scientific community's consciousness. With the steep rise of its downloads and its environment built on open-source packages, it is expected to become one of the most popular languages. This project introduces Julia, MDPs, the theory behind it, and possible approaches to its solution. By reviewing its drawbacks and advantages, we gradually get to present Finite-Horizon MDPs as the advanced solution. We design Finite-Horizon MDP Interface according to declaration at JuliaPOMDP/FiniteHorizonPOMDPs.jl \cite{FHPOMDP}, implement the interface and benchmark it against JuliaPOMDP's \cite{JuliaPOMDP} value iterator solver DiscreteValueIteration.jl \cite{DVI}.

\end{abstract-english}

% Acknowledgements / Podekovani
\begin{thanks}
We thank our Supervisor Ing. Jan Mrkos for all the little things he has done for us in making this possible, particularly for his patience. Our thanks also belong to our family and our girlfriend for the support they provided us with, and finally, to FEE CTU for all joys and sorrows it enriched us with.
\end{thanks}

% Declaration / Prohlaseni
\begin{declaration}
I declare that this work is all my own work and I have cited all sources I have
used in the bibliography.

\medskip

Prague, \monthinlanguage{title} \ctufield{day}, \ctufield{year}

\vspace*{2cm}

Prohlašuji, že jsem předloženou práci vypracoval samostatně, a že jsem uvedl veškerou použitou literaturu.

\medskip

V Praze, \ctufield{day}.~\monthinlanguage{second}~\ctufield{year}
\end{declaration}

% Only for testing purposes
\listfiles
\usepackage[pagewise]{lineno}
\usepackage{lipsum,blindtext}
\usepackage{mathrsfs} % provides \mathscr used in the ridiculous examples

\begin{document}

\maketitle

%!TEX ROOT=ctutest.tex

\chapter{Introduction}

MDP's are one of the most well-known methods used for solving stochastic planning problems. In the last couple of years, a new programming language, Julia, has emerged into the scientific community's consciousness. While offering a high-level approach comparable to the one of Python, it also offers a similar speed of languages like C or C++ at the same time. As the new users come to this language, the demand for new tools rises. The subject of this work answers these calls and deals with implementing one of those methods, finite-horizon MDPs interface for Julia's POMDPs.jl package.

\section{Markov Decision Processes}
The Markov decision processes are a part of the optimization problem solvers.
The term MDP was probably first used in \cite{cite:1} and was derived from the name of Russian mathematician Andrey Markov who researched stochastic processes, on which are MDPs based \cite{wiki:1}.
The problems that are MDPs solving are described with the environment. The environment consists of the states and actions (with corresponding stochastic transitions) for which the agent either receives a reward or pays cost (negative reward). The agent's task is to maximize his reward. The problem can be finite or infinite according to the problem's description.

\section{Julia}
Julia is a young language that starts to receive traction for its features. According to https://star-history.t9t.io/#JuliaLang\julia the number of stars on its repository(https://github.com/JuliaLang/julia) tripled over the last 20 months and is rising steadily.
The development started in 2009 at MIT, and the project went public in 2012 [https://www.zdnet.com/article/is-julia-fastest-growing-new-programming-language-stats-chart-rapid-rise-in-2018/]. 
It combines the performance of low-level languages like Fortran or C with the high-level approach of Matlab, R and Python. In addition to that, Julia is dynamically typed, designed for parallelism and distributed computation, it offers multiple dispatch paradigms and is also reproducible while relying on community packages[https://julialang.org/].


\section{POMDPs.jl}
\textit{POMDPs.jl is a package providing a core interface for working with MDPs and POMDPs}\cite{JMLR:v18:16-300}x It offers the user a unified interface for custom problem definitions, problem interfaces for simpler ones, and multiple solvers for both MDPs and POMDPs. The package is managed by Zachary Sunberg and is under active development.


\section{Goal}
The goal of this project is to implement a finite horizon interface for MDPs in the POMDPs.jl package. To get to know how the package's environment works, develop possible test cases, solve them using a custom value iteration method, and then compare them with the already implemented method in POMDPs.jl. Devise a way to implement finite horizon for MDPs in the package and evaluate its correctness. Get feedback from the POMDPs.jl community, iterate the design, and submit a pull request to the corresponding repository.




\chapter{Fully Observable Markov Decision Processes}

The Markov decision processes (MDPs) are a part of the optimization problem solvers. The term MDP was probably first used in \cite{cite:1} and came from the name of Russian mathematician Andrey Markov who researched the stochastic processes on which MDPs are based. The MDP problems are described with the environment. The environment consists of fully observable states and actions (with a corresponding stochastic transition model) for which the agent either pays a cost or receives a reward. The agent is an entity that lives inside of an MDP environment, and its task is to maximize the reward obtained for its actions. With the agent entity, we can intuitively demonstrate the MDPs solution. The environment does not change over time. The problem can be finite or infinite according to the problem's description.

This chapter will first draw a motivation on why we use MDPs in modeling environments in planning problems. Then define the background of MDPs, possible approaches to solving MDPs, and in the end, we are going to discuss the advantages of using time-dependent MDPs.

The definitions, algorithms, and the construction of the following sections are based on \cite{Kolobov2012}. 


% \section{Environment}
% The MDP environment we briefly introduced above contains states, actions, transition probabilities, and costs for each action. However, we did not assign any rules to it! 
% Without these rules, one can not create any connection between states and actions or actions and transition probabilities.

% The environmental rules could range anywhere from the ones applied in the crossword puzzle game (fully observable, deterministic, sequential, static, discrete with a single agent) up to the ones applied in the taxi driving planning (partially observable, stochastic, sequential, dynamic, continuous with multiple agents). We will not describe the meaning of all specific rules in this work, as it is out of its scope. In case the reader is interested, we refer to read  \cite{russel2010}.


% For the sake of simplicity, let us assume the following properties of the given environment:
% \begin{description}


%   \item[$\bullet$ Fully observable] The agent has all the information needed about the environment.
%   \item[$\bullet$ Stochastic] The movement of the agent is not certain. The result of movement can be different from the desired one as a result of noise,
%   \item[$\bullet$ Sequential] The current action is going to have an impact on future actions,
%   \item[$\bullet$ Static] The environment can not change over time,
%   \item[$\bullet$ Discrete] The environment has a finite number of states and actions.
% \end{description}



\section{MDP and its features}
% With the environmental properties sorted out, let us move on to MDPs themselves. 
% \\ 
% \\

The Fully Observable Markov Decision Processes (MDPs) are tuples \emph{(S, A, T, R)} containing a set of environment states $S$, actions $A$ that the agent can execute in corresponding states, stochastic transition function $T$ modeling the relationships between origin states $s$, actions $a$ from them and the distributions of the destination states $s'$. Finally, the MDPs also contain the reward function $R$ that assigns reward (or penalty) to tuples \emph{(s, a, s')} - for transitions from $s$ to $s'$ when executing the action $a$ More formally:


\begin{definition}[\textbf{Infinite Horizon Fully Observable Markov Decision Process (MDP)}]\label{def:MDP}
A fully observable MDP is a tuple $(S, A, T, R)$, where:
\begin{description}
  \item[$\bullet$ ] $S$ is the finite set of all possible states of the environment, also called the state space;
  \item[$\bullet$ ] $A$ is the finite set of all actions an agent can take;
%   \item[$\bullet$ ] $D$ is an infinite sequence of the natural numbers of the form $(1, 2, 3, \ldots)$, denoting the decision epochs, also called time steps, at which agent needs to act;
  \item[$\bullet$ ] $T : S \times A \times S \rightarrow [ \,0, 1] \,$ is a transition function mapping the probability $T(s, a, s')$ of reaching the state $s'$ if action $a$ is executed when the agent is in state $s$;
  \item[$\bullet$ ] $R : S \times A \times S \rightarrow R$ is a reward function that gives a finite numeric reward value $R(s, a, s')$ obtained when the system goes from state $s$ to state $s'$ as a result of executing action $a$.
\end{description}
\end{definition}

However, the MDP agent does not have free will. It simply moves as it is ordered. The agent's orders come from a policy. The policy is a function mapping every state to some action. Based on the agent's state, the agent executes an action corresponding to a policy's mapping of the agent's current state. The resulting quality of every policy is then a reward, which the agent receives for following the policy's mappings. In our context, the best quality is the maximal one.

Naturally, we want the agent to move in such a way that it maximizes its reward. Such a policy that is better than any other policy is called optimal policy. To select the optimal policy, we have to evaluate all possible policies. The number of possible policies $|A|^{|S|}$ exponentially grows with the increasing MDP state space. Furthermore, we do not consider separate time steps and different possible histories that could lead to a given state \textit{s} in a time \textit{t}. The history is a sequence ($s_0, a_0, s_1, a_1, \ldots, s_n$) of consequent states and actions that ends up in the current state $s_n$ by following a policy $\pi$. Such a formulation would inevitably rise to an uncountable number of policies.

Policies that do not distinguish between different histories that end up in the same state simultaneously are called Markovian policies. This relaxation in terminology vastly reduces the number of policies possible to $|A|^{|S|}$. In the following text, we are using the Markovian Policies.


% To obtain the maximum possible reward in the MDP environment, the agent must execute the best action for every state the agents get to. To determine which action is optimal in each state, we have to evaluate the execution of an action in each state. Say by using simulations. After an infinite number of simulation executions ending in terminal states, we get a vast number of state and action sequences with the value associated with them. The yet unknown mechanism creates a mapping that strives to create a sequence with the best reward from these sequences. Such a mapping of states to actions is called policy. Every agent then operates in its MDP environment by simply following the optimal policy. 

% However, the number of suboptimal policies obtained while selecting the optimal one is overwhelming, reaching the size of higher-order polynomials. Thus, we are going to define them in a more relaxed way.

% The Markovian Policy does not distinguish between two different sequences ending in the same state simultaneously, assigning the same best action to both, vastly reducing the number of policies.


\begin{definition}[\textbf{Markovian Policy}]
A probabilistic (deterministic) history-dependent policy $\pi: H \times A \rightarrow [ \,0, 1] \,(\pi: H \rightarrow A)$ is \textit{Markovian} if for any two histories $h_{s,t}$ and $h'_{s,t}$, both of which end in the same state $s$ at the same time step $t$, and for any action $a$, 
$\pi(h_{s,t}, a) = \pi(h'_{s,t}, a)$ $(\pi(h_{s,t}) = a$ if and only if $\pi(h_{s,t}) = a)$.
\end{definition}

Other than a whole policy, we can also evaluate a given state $s$. The value of a given state $s$ is calculated using a value function. The value function $V^{\pi}(s)$ returns the expected reward the agent obtains when executing the actions given by a policy $\pi$ from a state $s$ that end up in a terminal state. The terminal state stands for a sink state that the agent can not leave. The value function corresponds to $V: S \rightarrow [ \,-\infty, \infty] \,$. In the following text, we will refer to the value function as $V(s, t)$ or $V(s)$.


% Every state in every time step is evaluated according to \textbf{value function}. A Markovian value function corresponds to $V: S \rightarrow [ \,-\infty, \infty] \,$. In the following text, we will refer to the value function as $V(s, t)$ or $V(s)$.

% The value function of a policy is a value corresponding to executing actions according to a policy, obtaining rewards along the way. \\ \\

\begin{definition}[\textbf{The Value Function of a Policy}]
Let $h_{s,t}$ be a sequence that terminates at state $s$ and time $t$. Let $R^{\pi_{h_{s,t}}}_{t'}$ be random variables for the amount of reward obtained in an MDP as a result of executing policy $\pi$ starting in state $s$ for all time steps $t'$ s.t. $t \leqslant t' \leqslant |D|$ if the MDP ended up in state $s$ at time $t$ via sequence $h_{s,t}$.The value function $V^{\pi}: H \rightarrow [ \,−\infty,\infty] \,$ \textit{of a sequence-dependent policy} $\pi$ is a utility function $u$ of the reward sequence $R^{\pi_{h_{s,t}}}_t, R^{\pi_{h_{s,t}}}_{t+1}, \ldots$ that one can accumulate by executing $\pi$ at time steps $t, t + 1,\ldots$  after sequence $h_{s,t}$. Mathematically, $V^{\pi} (h_{s,t}) = u(R^{\pi_{h_{s,t}}}_t , R^{\pi_{h_{s,t}}}_{t+1}, \ldots)$.
\end{definition}


Among the policies evaluated with the value function $V^{\pi}$ from an initial state, we will be able to find an \textbf{optimal MDP solution} (or optimal policy) denoted as $\pi^*$ with value $V^*$ called the optimal value function. Such optimal solution then satisfies $V^{*} (h) \leqslant V^{\pi} (h)$ for all sequences $h \in H$.

% Among the policies evaluated according to the previous rule, we will be able to find an \textbf{optimal MDP solution} (or optimal policy) denoted as $\pi^*$ with value $V^*$ called the optimal value function. Such optimal solution then satisfies $V^{*} (h) \leqslant V^{\pi} (h)$ for all sequences $h \in H$.

One possible approach to evaluating the value function is using the expected sum of rewards from executing actions given by the policy, called the Expected Linear Additive Utility.
\\
\begin{definition}[\textbf{Expected Linear Additive Utility}]
An \textit{expected linear additive utility} function is a function $u(R_t, R_{t+1}, \ldots) = E[ \,\sum_{t'=t}^{|D|} \gamma ^{t'−t} R_{t'}] \, = E[ \,\sum_{t'=0}^{|D|-t} \gamma^{t'} R_{t'+t}] \,$ that computes the utility of a
reward sequence as the expected sum of (possibly discounted) rewards in this sequence, where $\gamma \geqslant 0$
is the discount factor.
\end{definition}

This approach eliminates the possibility of multiple different solution values resulting from multiple stochastic runs of the same policy. It employs the expected value and does not need to perform vector equality comparison as its results are scalars.

It turns out that this approach is even better as it guarantees a fundamental property of MDPs called \textbf{The Optimality Principle}, according to which \textit{among the policies that we evaluated by the expected linear additive utility, there exists a policy that is optimal at every time step.}

\section{MDPs techniques}

This part will briefly introduce a few fundamental algorithms for MDPs solving: Brute-Force algorithm, Policy Iteration, Value Iteration, and in the end, we present a possible solution for its disadvantages: Prioritizations. \\
After reading this section, the reader should know the options for solving MDPs and be ready for the following section to discuss the finite horizon's advantages.

\subsection{Brute-Force Algorithm}
As its name suggests, the Brute-Force algorithm is the most naive algorithm, similar to all problem-solving classes. During the execution of the solver, the method evaluates all possible output combinations and chooses the best one.
Moreover, computer scientists are not using it in the vast majority of cases because the algorithm needs to evaluate everything. In MDPs, the number of policies evaluated is $|A|^{|S|}$.

However, as the number of states or actions rises, such evaluation becomes enormous. Another problem is that action outcomes in MDPs are not deterministic, and MDP structure is often cyclic. Both of these problems result in a steep complexity rise in terms of computational resources.

That is why another approach comes in.

\subsection{An iterative approach to policy evaluation}

The evaluation of cyclic environments requires a suitable equation that captures all transitions and rewards. That means that every state's value function should correspond to the sum of rewards for moving towards successor states. Let us state the formula as follows:

% Moreover, the solver should output such a policy, whose value function for a given state maximizes its reward from future states. Thus, the successor state's value function multiplied by its probability should also appear in the equation. 

\begin{equation}
\begin{aligned}
V^{\pi} (s) = \sum_{s' \in S} T(s, \pi (s), s') [ \,R(s, \pi(s), s') + V^{\pi} (s')]
\end{aligned}
\end{equation}


% \begin{equation}
% \begin{aligned}
% V^{\pi} (s) & = 0 && \text{(if $s \in G$)} \\
% & = sum_{s' \in S} T(s, \pi (s), s') [ \,R(s, \pi(s), s') + V^{\pi} (s')] \, && \text{(otherwise)}
% \end{aligned}
% \end{equation}

For all states, the formula above forms a system of linear equations. This system of linear equations reduces the complexity of the previous approach. It \textit{can} be solved using Gaussian Elimination in $O(|S|^3)$. It is still not efficient enough, but we use the idea to find a solution with \textbf{an iterative approach}.

The iterative approach starts with a rough estimation and continuously works its way up to the asymptotically same solution as the linear system. The algorithm stops when the maximum gap between two consequent value functions of all states is lesser than a given value, called residual. This solution is optimal \cite{Kolobov2012}. As it stores the previous iteration, it is a part of dynamic programming. Every iteration of this algorithm runs in $O(|S|^2)$.

\LinesNumbered
\begin{algorithm}[ht]
\SetAlgoLined
//Assumption: $\pi$ is proper (ends up in a goal state) \\
initialize $V^{\pi}_0$ arbitrarily for each state \\
$n \xleftarrow{} 0$ \\
\Repeat{$ max_{s\in S} residual_n (s) \leq \epsilon$}{
$n \xleftarrow{} n + 1$ \\
\ForEach{$ s \in S $}{
compute $V^{\pi}_n (s) \xleftarrow{} \sum_{s' \in S} T (s, \pi (s), s') [ \,R(s, \pi (s), s') + V^{\pi}_{n−1} (s') ] \,$ \\
compute $residual_n (s) \xleftarrow{} |V^{\pi}_n (s) − V^{\pi}_{n−1} (s)| $ \\
}
}
return $V^{\pi}_n$
\caption{Iterative Policy Evaluation}
\end{algorithm}

We know how to evaluate the environment's state given some policy, but we have not yet described how to choose the best policy for a given evaluation. 

\subsection{Policy iteration}
Given that the iterative policy evaluation is optimal after an undefined number of iterations for a specific policy, we can further improve it by iterating policy evaluation and policy improvement.
For every iteration, we have \textit{almost} optimal policy evaluation for the previous policy. We can improve the policy by iterating over possible actions from all states and choosing that action if it has a greater reward than the previous one. The algorithm stops when the policy remains without a change after the following iteration. This algorithm is called Policy Iteration \ref{alg:PI}.

\LinesNumbered
\begin{algorithm}[ht]
\SetAlgoLined
initialize $\pi_0$ to be an arbitrary policy ending in the goal state\\
$n \xleftarrow{} 0$ \\
\Repeat{$\pi_n == \pi_{n−1}$ }{
    $n \xleftarrow{} n + 1$ \\
    Policy Evaluation: compute $V^{\pi_{n−1}}$ \\
    Policy Improvement: \\
    \ForEach{state $s \in S$}{
        $ \pi_n (s) \xleftarrow{} \pi_{n−1} (s) $ \\
        $ \forall a \in A$ compute $Q^{(V^{\pi_{n−1}})} (s, a) $ \\
        $ V_n (s) \xleftarrow{} max_{a \in A} Q^{(V^{\pi_{n−1}})} (s, a) $ \\
        \uIf{$ Q^{(V^{\pi_{n−1}})} (s, \pi_{n−1} (s)) > V_n (s)$}{
            $ \pi_n (s) \xleftarrow{} argmax_{a \in A} Q^{(V^{\pi_{n−1}})} (s, a)$
        }
    \textbf{end}
    }
}
return $\pi_n$
\caption{Policy Iteration}
\label{alg:PI}
\end{algorithm}

Until now, whenever we talked about policy iteration, we mentioned that the initial policy has to end in the goal state. If the policy ends up in the goal states, it converges significantly fast. On the other hand, if we do not meet the initial condition of policy ending in the terminal state, the policy evaluation step will diverge. Nevertheless, another algorithm can solve this drawback.


\subsection{Value iteration} \label{VI}
Value iteration focuses on improving the value function of all states instead of evaluating the policy. The algorithm creates a policy at its very end to maximize the reward in a given state. Thanks to this property, the initial policies are not used in the algorithm and can not lead to divergence.

The algorithm draws on from \textbf{Bellman equations}, which mathematically express the optimal solution of an MDP.

\begin{equation}
\begin{aligned}
V^* (s) & = max_{a \in A} Q^* (s, a) \\ 
Q^* (s, a) & = \sum_{s' \in S} T(s, a, s')[ \,R(s, a, s') + V^*(s')]  \\
\end{aligned}
\end{equation}

% \begin{equation}
% \begin{aligned}
% V^* (s) & = 0 && \text{(if $s \in G$)} \\
% & = max_{a \in A} Q^* (s, a) &&  (s \notin G) \\ 
% Q^* (s, a) & = \sum_{s' \in S} T(s, a, s')[ \,R(s, a, s') + V^*(s')] \, \\
% \end{aligned}
% \end{equation}

The equation's interpretation is maximizing the optimal value function of a given state. It can have a value of 0 if the state is among the goal states or a value-maximizing the result of performing a given action and then following the optimal policy if the state is not.

To approximate $V^* (s)$, the algorithm uses the so-called \textbf{Bellman backup} to improve the current $V_n (s)$ with $V_{n-1} (s')$.


\begin{equation}
\begin{aligned}
V_n (s) \xleftarrow{} max_{a \in A} \sum_{s' \in S} T(s, a, s') [ \,R(s, a, s') + V_{n - 1} (s')] \,
\end{aligned}
\end{equation}

% \begin{equation}
% \begin{aligned}
% V_n (s) \xleftarrow{} min_{a \in A} \sum_{s' \in S} T(s, a, s') [ \,R(s, a, s') + V_{n - 1} (s')] \,
% \end{aligned}
% \end{equation}

Value iteration is iteratively improving its value function estimation and is guaranteed to converge to the optimal solution \cite{Kolobov2012}. Its stopping criterion is either the maximum residual, the maximal difference between consecutive state value function estimations, or the number of iterations.

\LinesNumbered
\begin{algorithm}
\SetAlgoLined
initialize $V_0$ arbitrarily for each state \\
$n \xleftarrow{} 0$ \\ 
\Repeat{ $max_{s \in S} \: residual_n $ (s) < $ \epsilon $ }{
    $n \xleftarrow{} n + 1$ \\
    \ForEach{$ s \in S $ }{
        compute $V_n (s)$ using Bellman backup at $s$ \\ 
        compute $residual_n (s) = |V_n (s) - V_{n-1} (s)|$
        }
    }
return greedy policy: $\pi^{V_n} (s) = argmax_{a \in A} \sum_{s' \in S} \tau(s, a, s')[ \,C(s, a, s') + V_n (s')] \,$
\caption{Value Iteration}
\end{algorithm}


\subsection{Prioritization}
One of the most significant drawbacks of Value Iteration is that it requires full sweeps of the state space. This drawback results in many unnecessary value function evaluations because its value remains the same as the value function updates are heading from the goals state at first. It can take lots of iterations to evaluate all successor state's value functions. Furthermore, due to cyclic moves, the speed of convergence among states can differ.

Both of these problems often result in flawed performing algorithm execution.

Logical improvement is to, instead of iterating the whole state space, iterate over each state separately. Iterating over each separate states has two consequences:
\begin{enumerate}
    \item We have to develop or choose an algorithm that prioritizes states and 
    \item does not starve some of them (meaning every state gets updated accordingly).
\end{enumerate}

Furthermore, iterating over the separate states means that we can no longer use the number of iterations as a termination condition because we do not know which states' value functions and how often they were updated. The only terminal condition remains maximum residual.

\cite{Kolobov2012} formalizes the intuition in Algorithm \ref{alg:prior}:

\LinesNumbered
\begin{algorithm}[!ht]
\SetAlgoLined
initialize $V$ \\
initialize priority queue $q$ \\
\Repeat{ termination }{
    select state $s' = q.pop()$ \\
    compute V(s') using a Bellman backup at s'\\
    \ForEach{predecessor s of s', i.e. $\{s|\exists a [\, T(s, a, s') > 0] \,\}$}{
        compute $priority(s)$\\
        $q.push(s, priority(s))$\\
        }
    }
return greedy policy $\pi^{V}$
\caption{Prioritized Value Iteration}
\label{alg:prior}
\end{algorithm}

We will introduce a few priority metrics whose features generally differ and may diverge under specific conditions. For details, head over to \cite{Kolobov2012}.


\subsubsection{Prioritized Sweeping} 

Prioritized sweeping estimates the expected change in the value of a state if a backup were to be performed on it now.

\nopagebreak

\begin{equation}priority_{PS} (s) \xleftarrow{} max \Big\{ priority_{PS} (s), max_{a \in A} \big\{ T(s, a, s') Res^{V}(s')\big\} \Big\} \end{equation}


\subsubsection{Improved Prioritized Sweeping}

Improved Prioritized Sweeping employs the idea that states with fast converging value functions are to be prioritized. The consequence is that the algorithm first iterates the states near the goal and moves to other states only after the change between state iterations becomes small.

\nopagebreak

\begin{equation}priority_{IPS} (s) \xleftarrow{} \frac{Res^{V} (s)}{V (s)} \end{equation}

\subsubsection{Focused Dynamic Programming}

Focused Dynamic Programming is a particular case of prioritization techniques used when the start state $s_0$ is known. In such cases, the start case's knowledge can be employed and added as a penalty factor.


\begin{equation}priority_{FDP} \xleftarrow{} h(s) + V(s)\end{equation}


where:
\begin{description}[1cm]
  \item[h(s)] \textit{is lower bound on the expected reward for reaching s from $s_0$}, and
  \item[V(s)] is regular value function for given state s.
\end{description}

In the previous sections, we have very briefly introduced the notion of MDPs and their solving methods. First, we showed the Brute-Force algorithm, which naively evaluated all possible policies. Then we discussed the value and policy iterations that improved the performance but still did lots of useless evaluations. And finally, we presented a solution that intelligently iterates through prioritized states. 

\section{Finite-Horizon MDP} \label{FHMDP}

With all the necessary background layed out, Let us present the Finite-Horizon MDP.\\ \\

\begin{definition}[\textbf{Finite-Horizon MDP}]
A finite-horizon MDP is an MDP as described in ~\ref{def:MDP} with a finite number of time steps. %, i.e., with $|D| = T_{max} < \infty$.
\end{definition}


The Infinite-Horizon MDPs discussed so far are a slightly different class of problems from Finite-Horizon MDPs. However, the solution of Finite-Horizon is somewhat similar to the prioritized value iteration of Infinite-Horizon MDPs. 
Moreover, each Infinite-Horizon MDP can be transformed into Finite-Horizon one.

This is critical because the Finite-Horizon MDP has an acyclic state space, and \textbf{the acyclic MDPs can be solved optimally using only one backup if used optimal backup order.} \cite{Kolobov2012}

 On the other hand, the transformation to Finite-Horizon MDP %polynomially increases the state space by copying a whole former MDP into a single stage of Finite Horizon MDP. This way, the memory consumption linearly increases with the number of stages. 

Furthermore, the transformation to Finite Horizon MDP does not necessarily mean that the result will optimal or even the same. With a finite horizon, the finite horizon agents focus on getting rewards obtainable in a given time span. The Infinite Horizon MDPs do not have such a constraint and can focus on getting a bigger reward later on.

As the finite horizon number is known, we can evaluate all states' value functions from maximum horizon time and work our way down to starting time. This way, we evaluate each state's successor's value functions, employing optimal backup order and finding the optimal policy on the first pass.

In conclusion, in cases where the Infinite-Horizon methods are not sufficient, the MDPs can be transformed into Finite-Horizon ones. However, while solving the MDP in one pass, this transformation also blows up the state space.



\chapter{Implementation}


tady by se asi hodilo to rozdelit na jednotlive casti

\section{FiniteHorizonPOMDPs.jl}

jeste nevim kam presne zaradit interface, protoze to je pro MDP i pro POMDP

\section{FiniteHorizonDiscreteValueIteration.jl}






This chapter presents the solution design, implementation, and shows its benchmark.
As a goal for this project, we set to implement Finite-Horizon Solver for MDPs only.

\section{Solution Design}
The package's design is created in line with proposed declarations present in the original repository's README.md and discussed with Zachary Sunberg.

The goal of the project is to create a POMDPs.jl-compatible interface for defining MDPs and POMDPs with finite horizons, and in particular:
\begin{itemize}
    \item Provide a way for value-iteration-based algorithms to start at the final-stage and work backward
    \item Be compatible with generic POMDPs.jl solvers and simulators
    \item Provide a Finite-Horizon wrapper for Infinite-Horizon MDPs.
    \item Be compatible with other interface extensions like constrained POMDPs and mixed observability problems.
\end{itemize}

\subsection{Interface}
Moreover, the interface should contain or require the implementation of the following methods:
\begin{itemize}
    \item \textit{HorizonLength(::Type{<:Union{MDP,POMDP}}) = InfiniteHorizon()}
    \begin{itemize}
        \item \textit{FiniteHorizon}
        \item \textit{InfiniteHorizon}
    \end{itemize}
    \item \textit{horizon(m::Union{MDP,POMDP})::Int}
    \item \textit{stage\_states(m::Union{MDP,POMDP}, t::Int)}
    \item \textit{stage\_stateindex(m::Union{MDP,POMDP}, t::Int, s)}
    \item \textit{stage\_actions(m::Union{MDP,POMDP}, t::Int, [s])}
\end{itemize}

\subsection{Utilities}
And also the following utility:

\begin{itemize}
    \item \textit{fixhorizon(m::Union{MDP,POMDP}, T::Int) creates one of}
    \begin{itemize}
        \item \textit{FiniteHorizonMDP{S, A} <: MDP{Tuple{S,Int}, A}}
        \item \textit{FiniteHorizonPOMDP{S, A, O} <: POMDP{Tuple{S,Int}, A, O}}
    \end{itemize}
\end{itemize}

\section{Implementation}

The implementation is at \cite{FHPOMDP} or APENDIX.

The Finite Horizon algorithm evaluates a given MDP problem using Value Iteration.  

\subsection{Solution approach}

 The Current solution consists of a solver \textit{mysolve(mdp)} iterating and evaluating epochs and \textit{FiniteHorizonPolicy} structure storing its results (value function matrix, matrix of Qs, policy, map of actions, MDP instance). 
 
%  \textbf{Finite-Horizon Policy}
% \begin{description}[1cm]
%     \item $Q$ 
%     \item $V$  
%     \item $policy$ 
%     \item $actions$ 
%     \item $include\:Q$ 
%     \item $MDP$ 
% \end{description}
 
 \textit{Mysolve} takes the Finite-Horizon MDP as input, initializes policy structure and value function matrix, and iterates from \textit{$(maximum\_horizon-1)^{th}$} to \textit{$1^{st}$} epoch. Each epoch is then evaluated using a value iteration algorithm, which is initialized with the following epoch's value function matrix. Results of each epoch are stored in \textit{FiniteHorizonPolicy}.
 
 \LinesNumbered
\begin{algorithm}
\SetAlgoLined
initialize $FiniteHorizonPolicy$ \\
initialize $V$ \\
\For{$epoch = horizon - 1;\ epoch > 0;\ epoch = epoch - 1$}{
    set \textit{epoch} global \\
    run 1 iteration of \textit{value\_iteration(MDP, epoch, V)} \\
    update \textit{FiniteHorizonPolicy} \\
}
return \textit{FiniteHorizonPolicy}
\caption{Finite-Horizon MDP Solver mysolve}
\end{algorithm}

In the current version of the algorithm, we have to set the epoch global to pass it to interface functions. The next planned version will no longer contain the global variable.

\subsection{Defining MDP}

To define the MDP instance, we have to define the methods declared in \textit{JuliaPOMDP/POMDPs.jl} \cite{JMLR:v18:16-300} or from another library's package. The Finite-Horizon MDP Solver further requires: \textit{stage\_states}, which returns states of a given epoch, \textit{stage\_actions}, which return actions of a given epoch, and \textit{stage\_stateindex}, which return indices of a given state for a given epoch. For example of properly defined MDP see directory \textit{/test/instances}.
 

\subsection{Future plan}

The following releases' goals are to improve the performance, implement Infinite-Horizon MDP wrapper, and implement Finite-Horizon solver for POMDPs.


\section{Benchmark}
To prove the efficiency of implemented Finite-Horizon solver, we present a benchmark of Finite-Horizon MDP instance solved by
\begin{itemize}
    \item Value Iteration implemented in JuliaPOMDP/DiscreteValueIteration.jl
    \item Finite-Horizon Value Iteration solver implemented in this project in JuliaPOMDP/FiniteHorizonPOMDPs.jl package.
\end{itemize}

\subsection{MDP Instances}

To compare the results obtained, the instance has \textit{num\_of\_states * finite\_horizon} states for both solvers.

The instances were defined as 1D GridWorld problem with goal states on its right and left ends. The possible actions are \textit{move to left} and \textit{move to right}. The cost of such a move is 1, and the reward for moving to the goal is -10. Goals are terminal states and can not be left. The discount factor is one and the noise 0.6.

The benchmark were be performed on 3 different instance sizes:
\begin{itemize}
    \item \textbf{Instance 1} $states = 999, horizon = 500$,
    \item \textbf{Instance 2} $states = 99, horizon = 50$, and
    \item \textbf{Instance 3} $states = 9, horizon = 5$.
\end{itemize}

\subsection{Comparison}

The benchmark resulted in following:

\subsubsection{Instance 1}

\begin{itemize}
    \item \textbf{Value Iteration} 183.729947 seconds (3.48 G allocations: 244.793 GiB),
    \item \textbf{Finite-Horizon Value Iteration}   0.566122 seconds (7.17 M allocations: 579.764 MiB)
\end{itemize}

\subsubsection{Instance 2}

\begin{itemize}
    \item \textbf{Value Iteration} 1.082094 seconds (4.41 M allocations: 292.613 MiB)
    \item \textbf{Finite-Horizon Value Iteration}   0.199167 seconds (241.17 k allocations: 14.231 MiB)
\end{itemize}

\subsubsection{Instance 3}

\begin{itemize}
    \item \textbf{Value Iteration} 0.643639 seconds (1.08 M allocations: 51.450 MiB)
    \item \textbf{Finite-Horizon Value Iteration}   0.200691 seconds (172.53 k allocations: 8.653 MiB)
\end{itemize}  

We tested each benchmark's results to be the same.

The benchmarks show that the acyclic graphs with the optimal backup policy result in optimal policy and perform precisely both time-wise and memory-wise. We expect the results to be even better in the following versions.


\part{POMDPs}

\chapter{Theory}

Many real-world applications contain noisy feedback from its surroundings and as such their model does not fit into MDP representation. Having not deterministic stimulation results in a stochastic perception of the agent's state. Furthermore, the agent needs to store information about its possible location through the history of its stimulation. These stochastic features (although still covering only part of real-world models) demand a more general model for their representation - Partially Observable Markov Decision Problems. 

\section{Environment}
\begin{definition}[\textbf{Partially Observable Markov Decision Process (POMDP)}]\label{def:POMDP}
POMDPs \cite{Shani2013} are formally defined as a tuple $(S, A, T, R, \Omega, O, b_0)$, where:
\begin{description}
  \item[$\bullet$ ] \textit{S, A, T, R} are an \label{MDP} as defined above, often called the \textit{underlying} MDP of the POMDP.
  \item[$\bullet$ ] $\Omega$ is a set of possible observations. For example, in the robot navigation problem, $\Omega$ may
consist of all possible immediate wall configurations.
  \item[$\bullet$ ] \textit{O} is an observation function, where $O(a,s^t, o) = Pr(o_{t+1}|a_t,s_{t+1})$ is the probability of
observing \textit{o} given that the agent has executed action \textit{a}, reaching state \textit{$s^t$}. \textit{O} can model
robotic sensor noise, or the stochastic appearance of symptoms given a disease.
  \item[$\bullet$ ] $b_0$ is an initial belief distribution.
\end{description}

\end{definition}

\section{belief state and its updates}
Belief state \cite{Walraven19} is a $|S|$ long vector of probabilities $b(s)$ representing the distribution of presence in all states according to the agent's action and observation history.
One way to store information about the actual belief state distribution is through the history sequence as a tuple of a single action followed by a single observation for each time step. Such data storage requires additional memory to exist. However, this inconvenience can be removed by propagating the action and observation effects into already stored belief vectors. Thanks to the Markovian features and Bayes' rule, the belief state vector update can be expressed as following:
$$ b_o^a(s') = \dfrac{P(o|a, s')}{P(o|b, a)} \sum_{s\inS} P(s'|s, a) b(s),$$
where $P(o|b, a)$ corresponds to the probability to observe \textit{o} after executing action \textit{a} in belief \textit{b}. This probability is calculated as follows:
$$ P(o|b, a) = \sum{s'\inS}P(o|a, s')\sum{s\inS}P(s'|s, a)b(s).$$

This term serves as the normalizing constant. In [TODO: odkaz na BeliefUpdaters.jl], the normalizing constant is replaced with the norm of state vector update. Such solution is faster, but has no means to stop the execution if the probability is zero.

Thanks to its efficiency, the beliefs are used in the majority of POMDP solvers.


\section{POMDPs and its features}

\section{value function infinite horizon}

\section{value function finite horizon}

\section{solvers}

\subsection{approximate solvers}

\subsection{prunning}
jeste si nejsem jistej jestli ty dve predchozi kapitoly dam do implementace nebo do teorie


\chapter{Implementation}

This chapter is focused on the practical part of the thesis. IN the beginning, we present the assigned tasks connected with this paper. Then, we follow up with the analysis and description of the implemented methods. The sections after that present the benchmark problems and in the end, the validations and benchmarks are presented.

Assignment

The task of this paper is to survey and extend the methods implemented in the JuliaPOMDP library, with the emphasis on the finite horizon ones. To add any further methods or interfaces missing to implement such methods and to maintain the interoperability of the library. To evaluate the performance of the implemented methods and design new test instances, if applicable. And, if applicable, to benchmark the methods against other methods implemented in the library.


Design of the JuliaPOMDP library

Before we dive into the description of the tasks accomplished. It is necessary to introduce the JuliaPOMDP library itself.
The library is based on a set of interfaces gradually built on each other. Such foundations allow the community to create an environment in which the users can almost seamlessly change the functionality of their code. The interface offers uniform methods for defining problems (such as states for getting all states of the problem), structures for storing results in the policies (AlphaVectorPolicy for storing $\alpha$-vectors of POMDPs), passing the parameters to the solver(, or for executing the algorithm selected (calling `solve` is the same no matter the problem or the solver).

Methods defined in JuliaPOMDP

JuliaPOMDP's base method is the interface called POMDPs.jl. This interface defines all methods ranging from defining the model, through storing the result, to defining and executing solvers. On top of these methods, the user can either define a new problem or use already defined one in POMDPModels.jl repository. The user can also define a new solver, or use already defined one. Furthermore, the solvers return their results wrapped into a policy dedicated for it in POMDPPolicies.jl. If implementing a solver, the user would probably use the tools defined in POMDPModelTools.jl as, for example, an `ordered\_vector` which returns the wanted vector in an ordered manner, or distributions defined in the same repository.

ZMINIT ZE JE TO DIKY TOU INTEROPERABLE

The library also contains a lot of solvers for both MDPs and POMDPs. Ranging from Value Iteration up to Monte Carlo Tree Search solver for MDPs (4 total) and from approximate QMDP and other approximate solvers up to SARSOP or Incremental Prunning for POMDPs (12 total). Some of these solvers are reaching optimal solutions and some, usually approximate solvers, only suboptimal results, meaning that some are used more than the others. And they are also evaluated accordingly. However, with toal of 18 solvers (with 2 more reinforcement solvers) are offering more than enough to choose from when solving Infinite Horizon (PO)MDPs. 

Which concludes our analysis of the library.

The Finite Horizon POMDP solvers that are the main task of this thesis, are not yet included in the library. Moreover, the interface deemed essential for such solvers is not completed, as the solvers need to operate over single stages while the current interface does not support such handling.




ZMINIT, ZE PBVI NEFUNGOVALO


be, on whose foundations are the other methods implemented.






implement the others extend the JuliaPOMDP library with the functionality for solving finite horizon POMDPs. 
\section{solvers}

\subsection{approximate solvers}

\subsection{prunning}

jeste si nejsem jistej jestli ty dve predchozi kapitoly dam do implementace nebo do teorie

\section{PointBasedValueIteration.jl}
popsani toho jak to fungovalo predtim a jak jsem to upravil

\section{FiniteHorizonPointBasedValueIteration.jl}
popis implementace







Implementation

PBVI

The PBVI algorithm is stored in repo PointBasedValueIteration.jl and is partially based on former algorithm stored in the same place. The former algorithm, however, worker only for 2 states and completely omited the expansion phase, as the initial belief space was initiated with discrete distribution of beliefs.


The algorithm accepts problems defined in JuliaPOMDP/POMDPs.jl interface.



To some extent, the algorithm was written with list comprehensions or vectorizations, where it made sense. The utilization of vectorization was limited due to Julia's speed, often making the vectorizations slower than using for loops.



The PBVI's setting are handed over by the PBVISolver structure, which accepts following parameters:
num\_iteration, precision, verbose

The solve parameters are in line with POMDPs interface. That is, method solve accepts two parameters, solver's settings and POMDP to be solved. 


FIVI - TO BE COMPLETED 



















\input{ctutest-EX-problems}

\input{ctutest-EX-benchmark}

\chapter{Validation}

popis validaci a testu

porovnani s ostatnimi solvery, pouziti simulaci





\chapter{Optimizing and Profiling}

TOHLE PLATI O VSECH ALGORITHMECH? BUDE SAI LEPSI TO PREPSAT DO OPTIMALIZACI

To some extent, the algorithm was written with list comprehensions or vectorizations, where it made sense. The utilization of vectorization was limited due to Julia's speed, often making the vectorizations slower than using for loops.


OPTIMIZATIONS

The optimizations were done using Profiler.jl which records the bottlenecks in the algorithm.

It turns out that even if the algorithm itself is optimized, the biggest bottleneck are user-defined functions, if not optimized correctly. Furthermore, the user-defined POMDPs definition bottlenecks usually have the same bottleneck, that is - they return values in list. Such functions are allocation small memory blocks each time they are called, up to million times, resulting in massive overhead in memory consumption. Such allocation can be however easily fixed by replacing them with generator expressions. In Julia, this is achieved by replacing square brackets with round brackets.


Optimizing allocations from pushing to preallocations.
Optimizing allocations by precomputing array only once instead of each time in for loop.


\chapter{Conclusion}

This work deals with the contribution to open-source Julia framework \textit{JuliaPOMDP/POMDPs.jl} which combines interoperable interfaces on whose foundation users can implement their own (PO)MDP problem, solver, or any other tool. Users can also use any already existing tool the framework offers. Its task is to implement an interface for Finite Horizon (PO)MDPs, survey and analyze existing methods, and extend the framework with the new solver, emphasizing finite horizon ones.

In this work, we introduced all the necessary background for MDPs and POMDPs, both Infinite Horizon and Finite Horizon, as well as the motivation on why it is beneficial to use Finite Horizon (PO)MDPs in a time-constrained environment. On top of that knowledge, we implemented a missing interface for Finite Horizon (PO)MDPs called \textit{FiniteHorizonPOMDPs.jl} according to its design presented in the package template. With the interface set up, we implemented the \textit{fixhorizon} utility which wraps the Infinite Horizon (PO)MDPs into the Finite Horizon ones. After that, we modified the original Value Iteration algorithm to solve Finite Horizon (PO)MDPs, and in such a way that it serves as the example of a finite horizon solver. The Finite Horizon Value Iteration algorithm clearly builds on top of Infinite Horizon Value Iteration. It introduces the Finite Horizon solvers to an uninformed user in an easy-to-grasp way that demonstrates its structure and benefits.

During the survey of POMDP methods, we found out that the significant \textit{POMDPs.jl} solvers do not support \textit{FiniteHorizonPOMDPs.jl} interface and its modifications are not trivial. Furthermore, we found out that the implementation of \textit{PointBasedValueIteration.jl}, which is the generic solver for all Point-Based solvers, is not implemented correctly. As a result and a solution to a missing way of benchmarking the Finite Horizon POMDPs, we set up an additional task to implement the Point-Based Value Iteration as well. 

With the interface and benchmark solver implemented, we implemented Finite Horizon Point-Based Value Iteration. The FiVI is an excellent starting point for any future contributions to the finite horizon solvers of \textit{POMDPs.jl}. Furthermore, it demonstrates the necessity for specialized Finite Horizon solvers.

Other than previous solvers, this work contributes with additional POMDP models, policies, and other minor changes necessary to provide the Finite Horizon (PO)MDP support.

Finally, we provided benchmarks and experiments to validate our methods' correctness and their advantages. The finite horizon methods clearly show that its Infinite Horizon counterparts do not generalize, and the specialized solvers are essential. The PBVI shows that even that it is a generic solver without any clever heuristic, it still achieves similar results in the long run.

As future work, we are going to improve the support for Finite Horizon POMDPs. The main framework packages support Finite Horizon POMDPs, but there are still a few minor packages that do not, for example, \textit{POMDPSimulators.jl}. Other than that, Finite Horizon (PO)MDPs may well deserve a discrete belief designed for finite horizon specifically. Finally, on behalf of the JuliaPOMDP organization, any further solver contributions are encouraged. 


We believe that our framework addition will be of fair use to everyone out there searching for a way to solve their Finite-Horizon (PO)MDP problem with an already written package that is also efficient. The \textit{POMDPs.jl} is an excellently designed framework for POMDPs oriented tasks, and with the advent of Julia, it will get even more attention.


In conclusion, our work contributes to an extensive framework \textit{POMDPs.jl} and further improves its library environments. It introduces a whole new branch of problems and other tools not supported by \textit{POMDPs.jl} at the time of its completion. With the rise of Julia, it is the best time to contribute to Julian's open-source packages.



\appendix

% \printindex


\bibliographystyle{amsalpha}
\bibliography{ctutest}

% \ctutemplate{specification.as.chapter}

\end{document}